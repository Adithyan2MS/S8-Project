\chapter{LITERATURE SURVEY}
\thispagestyle{empty}
\\
\hspace{.2cm} PhishHaven—An Efficient Real-Time AI Phishing URLs Detection System [1] 2020 : Design a PhishHaven which detects and classifies a URL using three subcomponents.
First subcomponent, URL Hit 
The second subcomponent is Features Extractor.
The third subcomponent is Modelics.In this new paradigm executes ensemble-based machine learning models in parallel using multi-threading technique, and results in real-time detection by significant speed-up in the classification process. 

\hspace{.2cm}Sufficiency of Ensemble Machine Learning Methods for Phishing Websites Detection [2] 2021 : Phishing instances are usually derived from PhishTank
Other legitimate instances are from Alexa, DMOZ, and Common Crawl.
Features used in phishing detection are usually extracted from URLs (protocol, domain, path, parameters) This feature selection framework achieves a remarkable 87.6\% reduction in feature quantity with suffering from only a 0.1\% deterioration in detecting accuracy, making it possible for up-date training and real-time detecting in a production environment.

\hspace{.2cm} Eth-PSD: A Machine Learning-Based Phishing Scam Detection Approach in Ethereum
[3] 2021 :  Detect phishing scam-related transactions using a novel machine learning-based approach. Eth-PSD tackles some of the limitations in the existing works, such as the use of imbalanced datasets, complex feature engineering, and lower detection accuracy. Proposed Eth-PSD to detect the phishing scam in Ethereum. Started with derived requirements based on the limitations of related works and other effective IDSs from previous related works.


\hspace{.2cm} A Systematic Literature Review on Phishing Email Detection Using Natural Language Processing Techniques
[3] 2022 : The use of Natural Language Processing (NLP) techniques for detection of phishing except one that shed light on the use of NLP techniques for classification and training purposes, while exploring a few alternatives. In this research, journal, conference, and workshop papers were carefully analysed, published between 2006 and 2022, with different techniques to investigate the trend of phishing email detection.

\hspace{.2cm} OFS-NN: An Effective Phishing Websites Detection Model Based on Optimal Feature Selection and Neural Network
[4] 2019 :  In the proposed OFS-NN, a new index, feature validity value (FVV), is first introduced to evaluate the impact of sensitive features on the phishing websites detection. Then, based on the new FVV index, an algorithm is designed to select the optimal features from the phishing websites. This algorithm could properly deal with problems of big number of phishing sensitive features and the continuous changes of features. Consequently, it can mitigate the over-fitting problem of the neural network classifier.

\hspace{.2cm} Comparison of Classification Algorithms for Detection of Phishing Websites
[5] 2020 : Compare classic supervised machine learning algorithms on all publicly available phishing datasets with predefined features and to distinguish the best performing algorithm for solving the problem of phishing websites detection, regardless of a specific dataset design.
 

\hspace{.2cm} Phishing Detection using Random Forest, SVM and Neural Network with Backpropagation
[6] 2020 : The paper explains the improved Random Forest classification method, SVM classification algorithm and Neural Network with backpropagation classification methods which have been implemented with accuracies of 97.369\%, 97.451\% and 97.259\% respectively.


\hspace{.2cm} Intelligent rule-based phishing websites classification
[7] 2014 : The authors shed light on the important features that distinguish phishing websites from legitimate ones and assess how good rule-based data mining classification techniques are in predicting phishing websites and which classification technique is proven to be more reliable.The features extracted automatically without any intervention from the users using computerised developed tools.



\hspace{.2cm} Phishing sites detection based on Url Correlation.
[8] 2016 : Proposed Vulnerable Sites List and a new feature which is named URL Correlation. URL Correlation is based on the similarity of URLs with the List above that we created. A large improvement of accuracy is observed by comparing methods which use our new feature with the others which use the normal one.



\hspace{.2cm} Characteristics of Understanding URLs and Domain Names Features: The Detection of Phishing Websites With Machine Learning Methods [9] 2019 :The proposed method simplifies the process of feature extraction, and reduces processing overhead while going beyond analyzing on HTML, DOM, and URL based features by considering URLs, and domain names. A minimum loss in data conversion, selecting the appropriate machine learning technique, and consistency of definitions in the data set.



\hspace{.2cm} Detecting phishing websites using machine learning technique [10] 2018 : The proposed framework employs RNN—LSTM to identify the properties Pm and Pl in an order to declare an URL as malicious or legitimate. The proposed method (LURL) is developed in Python 3.0 with the support of Sci—Kit Learn and NUMPY packages. Also, the existing URL detectors are constructed for evaluating the performance of LURL. LURL has produced an average of 97.4\% and 96.8\% for Phishtank and Crawler datasets respectively.



\hspace{.2cm} Web Phishing Detection Using Machine Learning 
[11] 2022 : Suggest a literacy-based strategy to categorize Web sites into three
categories: benign, spam, and malicious. Our technology merely
examines the Uniform Resource Locator (URL) itself, not the
content of Web pages. As a result, it removes run-time stillness
and the risk of drug users being exposed to cyber surfer-based
vulnerabilities. When compared to a blacklisting service, our
approach performs better on generality and content since it uses
learning techniques.


\hspace{.2cm} Detection of Phishing Websites using Machine Learning [12] 2021 : Collect unstructured data of URLs from Phishtank website, Kaggle website and Alexa website, etc.
Train the three unique classifiers and analyse their presentation based on exactness two classifiers utilized are Decision Tree and Random Forest algorithm. Each classifier is trained using training set and testing set is used to evaluate performance of classifiers.
Performance of classifiers has been evaluated by calculating classifiers accuracy score.


 

\hspace{.2cm} A New Method for Detection of Phishing Websites: URL Detection
[13] 2018 : Random forest algorithm is implemented using Rstudio. The parsed dataset undergoes heuristic classification where the dataset is spilt into 70\% and 30\%. The 70\% data is considered for training and 30\% for testing. In this paper, a different methodology has been proposed to detect phishing websites by using random forests as the classification algorithm with the help of Rstudio.



\hspace{.2cm} Detection of URL based Phishing attacks Using Machine Learning [14] 2018 : Hybrid Algorithm Approach is a mixture of different classifiers working together which gives good prediction rate and improves the accuracy of the system. This system provides us with 85.5 \% of accuracy for XG Boost Classifier, 86.3\% accuracy for SVM Classifier, 80.2 \% accuracy for Naïve Bayes Classifier and finally 85.6 percentage of accuracy when using Stacking Classifier.



\hspace{.2cm} Large-Scale Automatic Classification of Phishing Pages 
[15] 2010 : We describe the design and performance characteristics of a scalable machine learning classifier we developed to detect phishing websites. We use this classifier to maintain Googles phishing blacklist automatically. Despite the noise in the training data, our classifier learns a robust model for identifying phishing pages which correctly classifies more than 90\% of phishing pages several weeks after training concludes. 
{\small
\begin{table}[H]
	\centering
	\caption{\textbf{Comparison between Related Works}}
	\label{table }
	\begin{tabular}{|p{.6cm}|p{3.5cm}|p{1.2cm}|p{1cm}|p{5.7cm}|}
		\hline
		Sl No  & Name of Paper &	Paper type & Year & Description \\
		
		\hline
		1&
		PhishHaven—An Efficient Real-Time AI Phishing URLs Detection System
		& 
		IEEE paper & 
		2020 &
		In this new paradigm executes
ensemble-based machine learning models in parallel using multi-threading tech-
nique, and results in real-time detection by significant speed-up in the classification
process.
		
		\\
		\hline
		2 &
		Sufficiency of Ensemble Machine Learning Methods for Phishing Websites Detection
		
		&
		IEEE paper &
		2021&
		This feature selection framework achieves a remarkable 87.6\% reduction in feature quantity with suffering from only a 0.1\% deterioration in detecting accuracy.
		\\
		\hline
		3&
		Eth-PSD: A Machine Learning-Based Phishing Scam Detection Approach in Ethereum
		&
		IEEE paper
		&
		2021&
  The project started with derived requirements based on the limitations of related works and other effective IDSs from previous related works.


		\\
		\hline
		4&
		OFS-NN: An Effective Phishing Websites Detection Model Based on Optimal Feature Selection and Neural Network
		&	
		IEEE paper
		&
		
		2019& In the proposed OFS-NN, a new index, feature validity value (FVV), is first introduced to evaluate the impact of sensitive features on the phishing websites detection
		\\
		\hline
  
		
		
		
		
	\end{tabular}
\end{table}
\newpage
\begin{table}
	\centering
	\label{table }
	\begin{tabular}{|p{.6cm}|p{3.5cm}|p{1.2cm}|p{1cm}|p{5.7cm}|}
		
		\hline
  
		5&
		Comparison of Classification Algorithms for Detection of Phishing Websites
		&
		IEEE paper
		&
		2020&
		The comparison results are presented in this paper, showing ensembles and neural networks outperforming other classical algorithms.
		\\
		\hline	
		6 &
		Phishing Detection using Random Forest, SVM and Neural Network with Backpropagation &
		IEEE paper
		&
		2020 &
		This paper explains the existing machine learning methods that are used to detect phishing websites. 

		\\
		\hline
		7 & Intelligent rule-based phishing websites classification

		& IET paper & 2014 & The features extracted automatically without any intervention from the users using computerised developed tools.
 \\
		\hline
		8 &Phishing sites detection based on Url Correlation
		&IEEE paper& 2016 & A large improvement of accuracy is observed by comparing methods which use our new feature with the others which use the normal one.
\\
		\hline
		9 &Characteristics of Understanding URLs and Domain Names Features: The Detection of Phishing Websites With Machine Learning Methods
		&IEEE paper& 2022 & A minimum loss in data conversion, selecting the appropriate machine learning technique, and consistency of definitions in the data set.

\\
		\hline
		10 &Detecting phishing websites using machine learning technique
  &IEEE paper& 2018 & The existing URL detectors are constructed for evaluating the performance of LURL. LURL has produced an average of 97.4\% and 96.8\% for Phishtank and Crawler datasets respectively.
\\
		\hline
		11 &Web Phishing Detection Using Machine Learning 
		&IJITEE paper& 2022 & Each classification is performed using a training set, and the performance of the classifiers is evaluated using a testing set. The accuracy score of classifiers was calculated to assess their performance.
\\
		\hline
	\end{tabular}
\end{table}
\newpage
\begin{table}
	\centering
	\label{table }
	\begin{tabular}{|p{.6cm}|p{3.5cm}|p{1.2cm}|p{1cm}|p{5.7cm}|}
		\hline
  
		12 &Detection of Phishing Websites using Machine Learning
		&IJERT paper& 2021 & Train the three unique classifiers and analyse their presentation based on exactness two classifiers utilized are Decision Tree and Random Forest algorithm.
\\
		\hline
		13 &A New Method for Detection of Phishing Websites: URL Detection
		&IEEE paper& 2018 & In this paper, a different methodology has been proposed to detect phishing websites by using random forests as the classification algorithm with the help of Rstudio.
\\
		\hline
		14 &Detection of URL based Phishing attacks Using Machine Learning 
		&IJERT paper& 2018 & This system provides us with 85.5 \% of accuracy for XG Boost Classifier, 86.3\% accuracy for SVM Classifier, 80.2 \% accuracy for Naïve Bayes Classifier and finally 85.6 percentage of accuracy when using Stacking Classifier.

\\
		\hline
		15 &Large-Scale Automatic Classification of Phishing Pages 
		&IEEE paper& 2010 & Despite the noise in the training data, our classifier learns a robust model for identifying phishing pages which correctly classifies more than 90\% of phishing pages several weeks after training concludes. 


\\
		\hline
		
	\end{tabular}
\end{table}
\\\\
}