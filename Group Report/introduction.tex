\chapter{INTRODUCTION}
\thispagestyle{empty}
\onehalfspacing
\pagestyle{fancy}
\fancyhf{}
\fancyhead[LE,RO]{\textit{\footnotesize \thepage}}
\fancyhead[RE,LO]{\textit{\footnotesize Phishing Detection Using Machine Learning}}
%\fancyfoot[LE,LO]{\textit{\footnotesize Department of CSE}}
\fancyfoot[LE,RO]{\textit{\footnotesize Department of CSE}}
 
\renewcommand{\headrulewidth}{2pt}
\renewcommand{\footrulewidth}{1pt}

Phishing  attack  is  a  simplest  way  to  obtain  sensitive information  from innocent  users. Aim  of the  phishers  is to acquire critical information like username, password and bank account details.  Cyber security  persons are now looking for trustworthy  and  steady  detection  techniques  for  phishing websites  detection. This  paper  deals with  machine learning technology for detection of phishing URLs by extracting and analyzing various features of legitimate and phishing URLs. Decision  Tree,  random  forest  and  Support  vector  machine algorithms are used to detect phishing  websites. Aim of the paper is to detect phishing URLs as  well as narrow down to best machine learning algorithm by comparing accuracy rate, false positive and false negative rate of each algorithm.



























































































































































